\documentclass[11pt,a4paper]{scrartcl}
\usepackage[utf8x]{inputenc}
\usepackage{ucs}
\usepackage{amsmath}
\usepackage{amsfonts}
\usepackage{amssymb}
\usepackage{graphicx}
\usepackage{listings}
\usepackage{tabularx}
\usepackage{listing}
\usepackage{xfrac}
\usepackage[german, ruled,noline]{algorithm2e}
\usepackage[bookmarks=true,unicode=true]{hyperref}

\hypersetup{pdftitle={Quadratwurzel},pdfauthor={Alexander Weigl}}

\lstset{postbreak=\space,breakindent=5pt,breaklines,basicstyle={\footnotesize},frame=tblr,numbers=left,float=tb}


\author{Alexander Weigl\\ \texttt{weigla@fh-trier.de}}
\title{Berechnung der Quadratwurzel $x^2 \equiv_n a$ im Fall $n$ ist prim oder $n=pq$ ein Primprodukt}

\begin{document}
\maketitle

\section{Einleitung}

Die Berechnung der Quadratwurzel von quadratischen Resten ist elementar in manchen kryptologischen Verschlüsselsystemen. Vorliegendes Werk dokumentiert die Erarbeitung einer Software für die Berechnung der Quadratwurzel $x$ zum quadratischen Rest $a$ aus der Gleichung:

\begin{equation}
						x^2 \equiv a \mod x \quad \text{mit } \operatorname{gcd}(a,x) = 1 \label{a}
\end{equation}

Eine solche existiert nicht immer (vgl. Tabelle \ref{tab:ex15}). Zu $a=5$ in $Z_{15}$  ist nicht relativ prim zu $15$. In diesen Fall bezeichnen wir $a$ als quadratischer Nichrest.

Für die Berechnung des quadratischen Restes fordere ich die Bedingungen 

\begin{align}
	n &\text{ ist prim }  &
	n  &= p\cdot 1 \text{ mit p,q sind prim } 
	\label{c}
\end{align}

Wenn $n=pq$ ein Primprodukt aus zwei Zahlen ist, kann man die Quadratwurzel über die Gleichungen: 

\begin{equation}
 x^2 = a \mod p \quad x^2 = a \mod p  \label{b}
\end{equation}

Damit ergibt sich die Forderungen nach der ersten Bedingung. 


\begin{table}
\label{tab:ex15}
\centering
\caption{Beispiel für Quadratwurzel in $Z_{15}$}
\begin{tabular}{r*{15}{|r}}
a & 0  &  1  & 2  & 3  & 4  & 5  &  6  &  7  &  8  & 9  & 10  & 11  & 12  &  13 & 14  \\ \hline
x & 0  &  1  & 4  & 9  & 1  & 10 &  6  &  4  &  4  & 6  & 10  &  1  &  9  &   4 &  1
\end{tabular}
\end{table}

\section{Grundlagen}
Im Folgenden möchte ich die Berechnung über einen Bottom-Up vorstellen.


\begin{quote}
	\textbf{1. Fall:}
	Sei $n$ prim. Dann kann die Lösung für die Gleichung \eqref{a} wie folgt bestimmt werden:
		\begin{itemize}
			\item $a^{\sfrac{(p-1)}{4}} \mod p$ falls $n \equiv 3 \mod 4$
			\item Ausprobieren aller $1 \le i \le \sfrac{n}{2}$ mit $i^2 =a \mod n$
		\end{itemize}
	
	In diesen Fall können wir überprüfen ob $a$ überhaupt ein quadratischer Rest ist, wenn die Gleichung  
	erfüllt wird:	
	\begin{equation}
		a^{\sfrac{(p-1)}{2}} \mod n = a
	\end{equation}
\end{quote}

\begin{quote}
	\textbf{2. Fall:}
		Sei $n=pq$ ist ein Primprodukt aus zwei Zahlen.
		Dann zerlegen wir in $n$ in seine Primfaktoren $pq$.
		Dann Lösen wir die Gleichungen \eqref{b} und erhalten vier Werte $x'_{pq_{12}}$.
		Mit den Gleichungen gehen wir in den chinesischen Restsatz:
		
		\begin{align}
			x_1 &\equiv x'_{p_1} \mod p & 	x_1 &\equiv x'_{q_1} \mod q \\
			x_2 &\equiv x'_{p_2} \mod p & 	x_2 &\equiv x'_{q_1} \mod q \\
			x_3 &\equiv x'_{p_1} \mod p & 	x_3 &\equiv x'_{q_2} \mod q \\
			x_4 &\equiv x'_{p_2} \mod p & 	x_4 &\equiv x'_{q_2} \mod q										
		\end{align}
		
		und erhalten $x_{1234}$ die unsere vier Lösungen für \eqref{a} darstellen.					
							
\end{quote}

\section{Umsetzung}

Der Ablauf für das Finden der Quadratwurzel wurde oben beschrieben. 
Es fehlen noch die Algorithmen für den Primzahlentest, Primfaktorzerlegung und den chin. Restsatz.

\subsection{Primzahlentest}

Der Primzahlentest entscheidet das Problem $p \stackrel{\in}{?} PRIM$. Für ein $p\in \mathcal{Z}$.
Wir verwenden dabei für $p<=10000$ einen Brute-Force-Ansatz indem durch alle Zahlen $2<=i<= \sqrt{p}$ 
geteilt $\mathcal{O}(2^n)$. Für $p>10000$ wird ein probalistier Ansatz verwendet, wenn $p\mod 2=1$:

\begin{algorithm}[H]
	Wähle $a_1, \ldots, a_k$ zufällig aus $\{1, \ldots , p-1\}$\;
	\ForEach{$a_i$}{
			$t=a_{i}^{\frac{p-1}{2}}$\;
			\If{not $t \equiv \pm 1 \mod p$}{
				$p \notin PRIM$\;
				break
			}
	}
\end{algorithm}

Die Fehlerwahrscheinlichkeit sinkt mit größeren $k$: $ \le \sfrac{1}{2}^k$. Die Implementierung ist in Listing \ref{lst:primtest}. Hier wird das $k$ anhand der gg. Fehlertoleranz bestimmt.


\subsection{Primfaktorzerlegung}

Die Zerlegung von $n$ in die Primfaktoren ist relativ einfach, wenn man die Primzahlen von 
$1,\ldots ,\sfrac{n}{2}$ kennt. Wir können nun die Teilbarkeit von $n$ und den Primzahlen prüfen.
Es gibt ungeführ $\frac{n}{2 \ln n - ln 2}$ solcher Primzahlen gibt. Das Finden der Primzahlen erfolgt über das Sieb des Erosthenes:

\begin{algorithm}[H]
\caption{Finden aller Primzahlen bis $n$}
\KwData{$a_1, \ldots a_n ~,~ a_i \in \{0,1\}$}
\KwResult{$a_i = 1 \rightarrow i \text{ ist prim}$}
Initialisierung $a_i=1$\;

$a_{01}=0$\;

\For{$i := 2$ \KwTo $n$}{
		\If{$a_i = 1$}{
			\For{$j:=2\cdot i$ \KwTo $n$}{
				$a_j=0$
			}	
		}		
	}
\end{algorithm}

Der Algorithmus markiert alle alle Zahlen, die ein Vielfaches einer Primzahl sind. 
Der Nachteil des Siebes ist ein höher Speichernplatzverbrauch. Die Implementierung des Siebes ist im Listing \ref{lst:sieb}; die des Primfaktorzerlegung in \ref{lst:faktor}.

\section{Programmvorstellung}

Die Entwicklung des Programmes erfolgte in Java und wurde später mit GWT\footnote{Google Web Toolkit: \url{http://google.de/?q=GWT}} in Javascript umgewandelt. Die Verwendung von GWT hat den Vorteil der statischen Prüfung der Syntax und Methodensignaturen.

In der Eingabe Maske kann des $a$ und $n$ in der Form der Gleichung \eqref{a} erfolgen. 
Ergebnisse werden anschließend in der unterhalb dargestellt. Dabei versucht die Applikation 
alle Berechnung über Proberechnungen zu verfizieren. Sollte es nicht möglich sein die Quadratwurzel zu berechnen wird der Prozess mit einer \texttt{NoSolutionException} abgebrochen

\section{Fazit}
Die Fällen \eqref{c} in werden unter der Bedingung das die Zahlen in das \texttt{Integer} Datentyp passen erfolgreich berechnet. Das Limit auf den  Datentyp \texttt{Integer} ruht von der maximalen Größe von Arrays in Java wieder, die beim Sieb des Erosthenes eingesetzt werden. 

Im Weiteren gibt es noch einen Sonderfall für \eqref{a}: bei $p\equiv 1 \mod 4$. Dieser Fall kann in einer späteren Funktion ergänzt werden\footnote{Fallunterscheidung in \texttt{rootWithPrime}}.


\begin{center}
 \href{http://creativecommons.org/licenses/by-nc-sa/3.0/de/legalcode}{	License: Creative Commons v3.0 - by-nc-sa} \\
 Source Code: \url{http://github.com/areku/QuadraticRemainderRoot}
\end{center}

\hrulefill
\section{Listings}

\begin{lstlisting}[language=Java,caption={Sieb des Erosthenes},label=lst:sieb]
public static List<Integer> getPrimes(int n) {
	LinkedList<Integer> primes = new LinkedList<Integer>();
	boolean b[] = new boolean[n];
	for (int i = 2; i < n; i++) {
	    if (!b[i]) {
			primes.add(i);
			for (int j = i; j < n; j += i) {
			    b[i] = true;
			}
		}
	}
	return primes;
}
\end{lstlisting}

\begin{lstlisting}[language=Java,caption={Primfaktorzerlegung},label=lst:faktor]
public static int[] factors(int n) {
	List<Integer> factors = new LinkedList<Integer>();
	List<Integer> primes = getPrimes((int) (n / 2));
	long num = n;

	Iterator<Integer> iter = primes.iterator();
 	while (num > 1 && iter.hasNext()) {
	    int p = iter.next();
	    if (num % p == 0) {
		factors.add(p);
		num = num / p;
	    }
	}

	int[] l = new int[factors.size()];
	int i = 0;
	for (int m : factors) {
	    l[i++] = m;
	}
	return l;
}
\end{lstlisting}

\begin{lstlisting}[language=Java,caption={GCD}]
 public static int chineseRemainder(int[] a, int m[]) {
	if (a.length != m.length) {
	    throw new IllegalArgumentException();
	}

	int n = m.length;

	long product = 1;
	for (int i = 0; i < n; i++) {
	    product *= m[i];
	}

	long M[] = new long[n];
	for (int i = 0; i < n; i++) {
	    M[i] = product / m[i];
	}

	long Minv[] = new long[n];
	for (int i = 0; i < n; i++) {
	    Minv[i] = GCD.extendedGCDFactorFirst(M[i], m[i]);
	}

	int x = 0;
	for (int i = 0; i < Minv.length; i++) {
	    x += (a[i] * M[i] * Minv[i]) % product;
	}
	
	x%=product;
	
	if(x<0)
	{
	    x+=product;
	}
	return x;
}
\end{lstlisting}

\begin{lstlisting}[language=Java,caption={statistischer Primzahlentest},label={lst:primtest}]
	public static boolean statTest(int n, double prob) {
	if ((n & 1) == 0) {
	    return false;
	}

	if (n <= 10000)// kleiner Bruteforce
	{
	    for (int i = 3; i < Math.sqrt(n); i++) {
		if (n % i == 0)
		    return false;
	    }
	    return true;
	}

	int k = 1;
	double p = 0.5;
	for (; k < 23; k++) {// Ueberschreitung maschinengenauigkeit
	    if (p <= prob)
		break;
	    p /= 2;
	}

	Random r = new Random();
	for (int i = 0; i < k; i++) {
	    int ai = r.nextInt((int) n - 1);
	    long t = (long) (Math.pow(ai, (n - 1) / 2)) % n;
	    if (t == 1)
		return true;
	}
	return false;
    }
\end{lstlisting}


\end{document}
